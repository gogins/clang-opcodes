%
%-----------------------------------------------------------
%% Computer Music Journal LaTeX template
%%
%% September  2009
%% Author: Cornelia Kreutzer, University of Limerick

% Maximum of 32 pages in the PDF including figures and references.
% Present first abstractly, then illustrate concretely, then 
% discuss examples. 
% Include figures for Clang/LLVM, for ORC, and for the Csound 
% opcodes.

%---Document preamble
%
\documentclass[letterpaper, 12pt]{article}


\usepackage{cmjStyle} %use CMJ style
\usepackage{natbib} %natbib package, necessary for customized cmj BibTeX style
\bibpunct{(}{)}{;}{a}{}{, } %adapt style of references in text
\doublespacing
\raggedright % use this to remove spacing and hyphenation oddities
\setlength{\parskip}{2ex}
\parindent 24pt
\urlstyle{same} % make url tags have the same font
\setcounter{secnumdepth}{-1} % remove section numbering
\usepackage{epstopdf}
\usepackage{amsmath,amssymb,amsbsy,bm,upgreek,nicefrac}
\usepackage{todonotes,microtype}

% Use the Figures subfolder for image files
\graphicspath{{./Figures/}}


%% ----------------------------------------------------------------------------------------------------------------------------------------
%% CMJ page headers
%% For initial submission use \lhead{Anonymous}
%% On acceptance for publication, use real author surnames for \lhead modeled on the following examples
%%		One author:	\lhead{\small Keislar}
%%		Two authors:	\lhead{\small Keislar and Castine}
%%		Three authors:	\lhead{\small Keislar, Castine, and Rundall}
%%		Four or more:	\lhead{\small Keislar et al.}
%%
\lhead{\small Anonymous}


%% The package endfloat moves all floats (figures, tables...) to the end of the article, as required for the final version of a CMJ article.
%% Leave this package commented out for initial submission, but uncomment it and the following callout commands for the final version. 
% \usepackage{endfloat}
% \renewcommand{\figureplace}{%
%	\begin{center}
%		\textbf{<<TYPE: INSERT \figurename~\thepostfig\ ABOUT HERE.>>}
%	\end{center}}
% \renewcommand{\tableplace}{%
%	\begin{center}
%		\textbf{<<TYPE: INSERT \tablename~\theposttbl\ ABOUT HERE.>>}
%	\end{center}}

%---Document----------
\begin{document}

{\cmjTitle Embedding a Runtime C++ Compiler in a Software Sound Synthesis System}
\vspace*{24pt}

(In the initial submission, omit all the following author information to ensure anonymity during peer review.
On final submission please make sure that the author address is a complete, functioning postal address.
Post will be sent to that address.)

% Author: name
{\cmjAuthor Firstname Lastname}	% List all authors here
							% e.g.:
							% {\cmjAuthor Doug Keislar, Peter Castine, and Jake Rundall}
 
% Author: address
\begin{cmjAuthorAddress}
	Sound Computing Group Full Address\\
	University of Anywhere\\
	1234 Anywhere Street\\
	Somecity, Somestate 012345 USA\\		% Adapt as needed for non-US addresses
	email@email.com
\end{cmjAuthorAddress}


\begin{abstract}
	This article considers why musicians and researchers might want to use a C++ compiler embedded in a software synthesis (SWSS) system, and how such a compiler can be implemented and used, by presenting new opcodes for the Csound SWSS that embed the Clang/LLVM on-request compiler (ORC). C++ source code may be embedded in a regular Csound orchestra file, compiled by the \texttt{Clang\_compile} opcode, and invoked during performance by the \texttt{Clang\_invoke} opcode. Uses include writing new signal processing and synthesis code in C++ right in the SWSS, full-strength algorithmic composition right in the SWSS, calling into external dynamic link libraries right from the SWSS, creating native user interfaces for a piece right in the SWSS, and more. The technology and patterns presented here could be adapted for use in any SWSS that supports the C calling convention. 
\end{abstract}

\section{<<BEGIN ARTICLE>>}
% Why do it?
Let us begin with the question, \textit{Why do this?} Basically, to provide more computer power while, at the same time, speeding up the musician's workflow. In computer music, and in particular in algorithmic composition, there is generally a work cycle that goes something like this:

\begin{enumerate}
\item \textit{Coding time}: write or edit some source code.
\item \textit{Building time}: compile the source code.
\item \textit{Debugging time}: if the code doesn't run, debug it, then go back to 1.
\item \textit{Composition time}: Use the software to actually write some music.
\item \textit{Rendering time}: Use the software to actually perform the music.
\item \textit{Audition time}: Critically listen to the music. If you are not satisfied, go back to step 1, 2, 3, 4, or 5.
\end{enumerate}

\noindent At the birth of computer music, each of these steps was agonizingly slow. As computers have increased in memory and speed, the steps have certainly gotten faster and faster. Today, for most pieces, it is possible to render a piece to real-time audio, and thus to collapse steps 5 and 6 above:

\begin{enumerate}
\item \textit{Coding time}: write or edit some source code.
\item \textit{Building time}: compile the source code.
\item \textit{Debugging time}: if the code doesn't build or run, debug it, then go back to 1.
\item \textit{Composition time}: Use the software to actually write some music.
\item \textit{Audition time}: Use the software to actually render the music to real-time audio, and listen to it critically. If you are not satisfied, go back to steps 1, 2, 3, or 4.
\end{enumerate}

Today also, a new paradigm for coding and composing is emerging. This involves the use of a SWSS that is also a development system. Sometimes this can mean using a SWSS that is designed as an all-in-one integrated development system, such as SuperCollider \citep{supercollider, mccartney2002rethinking}. Sometimes this can mean using the SWSS from a dynamic language, as when Csound \citep{csoundmain, lazzarini2016csound, csoundbook} is used from Python \citep{python} and the piece also is composed in Python using a library such as musx \citep{musx} or CsoundAC \citep{csoundextended}.

And sometimes, and this is perhaps more interesting, it can mean embedding a development system in the SWSS, as when the Faust language \citep{faust, orlarey2009faust} is embedded in Csound using the Faust opcodes \citep{Lazzarini2014, faustcompile, faustdsp}. This is what we present here: new Clang opcodes for Csound \citep{clangpcodes}. The user writes C++ code right in the Csound orchestra, the Clang opcodes automatically compile that source code at performance time, and the compiled code runs as part of the Csound performance... at the speed of native C++.

The effect then is to collapse steps 1, 2, 3, and 4 above, with this result:

\begin{enumerate}
\item \textit{Composition time}: write or edit some source code that generates and/or renders a piece.
\item \textit{Audition time}: Use the software to actually render the music to real-time audio, and listen to it critically. If the piece fails to compile or play, go back to step 1 and debug it. If the piece plays but is not musically satisfactory, go back to step 1 and edit it.
\end{enumerate}

\indent In this way, it is possible to move \textit{all} of the work involved in composing \textit{and} programming computer music into the composition time and audition time parts of the work cycle.

And that is as it should be.

\section{The Runtime Compiler}

A \textit{runtime compiler} is a computer that translates source code into executable form during the run time of some host program without requiring the user of a build system or external programs that preprocess, compile, and link the code. For a dynamic language such as Python or, for that matter, Csound, the host program is the compiler itself. Yes, any SWSS that translates source code or user-defined patches into sound without external assistance is a runtime compiler.

Until recently, although runtime compilers and interpreters for C and C++ certainly did exist, thanks to numerous issues they were not generally used. With the advent of the low-level virtual machine (LLVM) and Clang projects \citep{llvm}, that has changed. The primary motive of Clang/LLVM is to provide a modular system of dynamic link libraries that implement a drop-in replacement for the widely used GNU Compiler Collecton (GCC) \citep{gcc}. But the modular design of LLVM/Clang greatly facilities the implementation of runtime compilers either for domain-specific languages (as with the Faust DSP system), or for general-purpose C++ (as with the Clang opcodes presented here).

On the most basic level, the Clang/LLVM system works as follows:

\begin{enumerate}
\item \textit{Front end}: translate source code to the machine language (called intermediate representation (IR)) for an abstract, low-level virtual machine (LLVM). Each translation unit becomes a module of IR.
\item \textit{Back end}: translate modules of IR to modules of actual exectuable machine language, link them, relocate them, and resolve all symbols. The modules become an executable program or dynamic link library.
\end{enumerate}

\noindent The beauty of the LLVM system is that a new front end can fairly easily be added to the system (because all front ends emit the same IR). For example, the Faust language \citep{faust} includes its own new LLVM front end that translates Faust source code to IR. Similarly, a new back end can easily be written for any runtime architecture. The only code that needs to be written for a new runtime is the code that emits native machine language instructions for each IR instruction. In theory, any front end (e.g. Clang) will work with any back end (to run on Windows, the macOS, Linux, WebAssembly, etc., etc.).

\begin{figure}[]
\begin{center}
\includegraphics[width=\textwidth,height=\textheight,keepaspectratio]{CLangLLVMArchitecture}\caption{Overview of LLVM Architecture.}
\label{fig:llvm}
\end{center}
\end{figure}

For a \textit{runtime} compiler, the back end is modified to provide for emitting, loading, relocating, and linking native machine language at run time. The Clang/LLVM system that does this is called the on-request compiler (ORC) \citep{llvmorc}. It is a type of just-in-time (JIT) compiler that emits native machine language for a symbol the very first time the address of that symbol is requested from the LLVM execution session. The ORC compiler and LLVM can load and link external dynamic link libraries --- whether system libraries, or user libraries, or LLVM's own in-memory JITDylib libraries --- at run time, and thus perform the functions of a build system linker or an operating system's linking loader. 

In Figure \ref{fig:llvm}, the components involved in the Csound opcodes are labeled in boldface italics. The components used by the standard, command-line toolchain include a linker and are connected with dotted arrows. If the opcodes are compiled for another architecture such as ARM, the ORC compiler and the LLVM context will adapt and emit the correct code for that architecture.

As an example of how to embed a runtime C++ compiler into a SWSS, we present two new opcodes for Csound: \texttt{clang\_compile} and \texttt{clang\_invoke}. The Clang/LLVM code used in these opcodes is adapted and expanded from the "clang-interpreter" example in the LLVM Project's GitHub repository \citep{clanginterpreter}. The high-level design is similar to Csound's Faust opcodes \citep{Lazzarini2014}: one opcode to compile source code (e.g. \texttt{faustcompile} \citep{faustcompile}, \texttt{clang\_compile}), another opcode to invoke compiled code (e.g. \texttt{fuastdsp} \citep{faustdsp}, \texttt{clang\_invoke}). Figure \ref{fig:clangopcodes} shows how these opcodes relate to the Clang/LLVM components.

\begin{figure}[]
\begin{center}
\includegraphics[width=\textwidth,height=\textheight,keepaspectratio]{EmbeddedCLangLLVMArchitecture}\caption{Overview of Csound Clang Opcodes.}
\label{fig:clangopcodes}
\end{center}
\end{figure}

\section{The Clang Opcodes for Csound}

All instances of both Clang opcodes share one global instance of the ORC compiler, that is, the back end of the Clang/LLVM system. However, the \verb|clang_compile| opcode sets up the front end of the Clang/LLVM system anew for each module.

\subsection{clang\_compile}

\begin{Verbatim}[fontfamily=courier, xleftmargin=\parindent]
i_result clang_compile S_code, S_options [, S_libraries]
\end{Verbatim}

\begin{description}
\item[\texttt{S\_code}] This is the C++ source code to be compiled. It can be a multi-line string literal enclosed in \verb|{{| and \verb|}}|. Any slashes in the code must be escaped, i.e.\ for a newline use \verb|\\n| and not \verb|\n|. This code must declare and define a 
uniquely named entry point function with the signature:

\begin{Verbatim}[fontfamily=courier, xleftmargin=\parindent]
extern "C" int (*)(CSOUND *csound);
\end{Verbatim}

\item[\texttt{S\_options}] This is any number of standard compiler options, delimited by spaces, as would be given to gcc or clang. It can be a multi-line string literal enclosed in \verb|{{| and \verb|}}|. If there are no custom compiler options, an empty string must be passed. If the options include \verb|-v|, not only are Clang diagnostics printed, but also additional diagnostics from the Clang opcodes. It is recommended to pass \verb|-march=native| to take advantage of all possible LLVM optimizations for the target architecture.
\item[\texttt{S\_libraries}] This parameter is optional and can be any number of dynamic link libraries given as fully specified filepaths separated by spaces. The parameter can be a multi-line string literal enclosed in \verb|{{| and \verb|}}|. The standard \texttt{-l} compiler option does not work in this context.
\item[\texttt{i\_result}] is returned as 0 for success and non-0 for failure. 
\end{description}

The \verb|clang_compile| opcode runs at Csound's initialization time, which 
comes after \verb|csoundStart| has been called. If the opcode is called from the 
orchestra header, i.e. in \verb|instr 0|, then compilation and execution is done during Csound's init pass for \verb|instr 0|. If the opcode is called from a regular Csound 
instrument, then compilation and execution is done during Csound's init pass for that particular instrument instance.

Once \verb|clang_compile| has compiled a module, Csound will immediately 
call the entry point function in that module. At that very time, the LLVM ORC 
compiler will translate the IR code in the module to machine language, perform 
relocations, resolve symbols, and otherwise load and link the module into the 
running Csound process, just like any other C++ module.

The entry point function may call any Csound API functions that are members of 
the \verb|CSOUND| struct, define classes and structs, call any public symbol in any 
loaded dynamic link library, or indeed do anything at all that can be done 
using C++ code.

For example, the module may use an external shared library to assist with 
algorithmic composition, then translate the generated score to a Csound score, 
then call \verb|csound->InputMessage| to schedule that score for immediate 
performance.

However, one of the most significant uses of \verb|clang_compile| is to compile C++
code into classes that can perform the work of Csound opcodes. This is 
done by implementing the \verb|ClangInvokable| interface, as described below.

\subsection{clang\_invoke}

\begin{Verbatim}[fontfamily=courier, xleftmargin=\parindent]
[m_out_1,...] clang_invoke S_invokeable, i_thread [, m_in_1,...]
\end{Verbatim}

\begin{description}
\item[\texttt{S\_invokable}] This is the unique name of a factory function with signature \verb|ClangInvokable *(*);| in the compiled modules that returns a new instance of an object that implements the pure virtual \verb|ClangInvokable| interface:

\begin{Verbatim}[fontfamily=courier, xleftmargin=\parindent]
struct ClangInvokable {
	virtual ~ClangInvokable{};
	virtual int init(CSOUND *csound, OPDS *opds, 
		MYFLT **outs, MYFLT **ins) = 0;
	virtual int kontrol(CSOUND *csound, 
		MYFLT **outs, MYFLT **outs) = 0;
	virtual int noteoff(CSOUND *csound) = 0;
};
\end{Verbatim}

\item[\texttt{i\_thread}] This must be 1 if the \verb|ClangInvokable::init| method will be called once during the lifetime of the containing instrument instance, but not the \verb|ClangInvokable::kontrol| method; 2 if if the \verb|ClangInvokable::init| method will not be called but the \verb|ClangInvokable::kontrol| will be called once per kperiod during the lifetime of the containing instrument instance; and 3 if the \verb|ClangInvokable::init| method will be called once and then the \verb|ClangInvokable::kontrol| method will be called once per kperiod during the lifetime of the containing instrument instance.
\item[\texttt{[m\_in1, m\_in2,...}] are 0 or more \verb|MYFLT **ins| arguments to the \verb|ClangInvokable| methods. They correspond exactly to the types, sizes, shapes, and rates of the parameters (following the \verb|CSOUND| pointer and the \verb|OPDS| pointer) prepared by Csound to call the \verb|clang_invoke| opcode, which in turn will call \verb|ClangInvokable|. These can be scalar floats, a-rate arrays of floats, strings, streaming phase vocoder frames, or any other data type as specified for use by opcodes of the Csound orchestra language in the \verb|entry1.c| \citep{entry1} file.
\item[\texttt{[m\_out1, m\_out2,...}] are 0 or more \verb|MYFLT **outs| arguments to the \verb|ClangInvokable| methods. They correspond exactly to the types, sizes, shapes, and rates of the parameters returned to Csound by the \verb|clang_invoke| opcode, which in turn has obtained them from the \verb|ClangInvokable| outputs. These again can be scalar floats, a-rate arrays of floats, strings, streaming phase vocoder frames, or any other data type as specified for use by opcodes of the Csound orchestra language in the \verb|entry1.c| \citep{entry1} file.
\end{description}

\noindent The enormous flexibility yet simplicity of the interface between Csound and its opcodes, still using the original design of Barry Vercoe, has greatly simplified the design of the Clang opcodes.

The \verb|S_invokeable| symbol is looked up in the LLVM execution session 
of the global ORC compiler, and a new instance of the \verb|ClangInvokable| class 
is created. 

\verb|clang_invoke| then calls the \verb|ClangInvokable::init| method with 
the input and output parameters, and any values computed by the 
\verb|ClangInvokable| are returned in the elements of the \verb|outs| argument.

If the \verb|i_thread| parameter is 2 or 3, the \verb|ClangInvokable::kontrol| method is 
called once per kperiod during the lifetime of the opcode. Any values 
computed by the \verb|ClangInvokable| are returned in the elements of the \verb|outs| argument. 

The \verb|clang_invoke| opcode is very efficient. The code for a \verb|ClangInvokable| class is compiled the very first time it is accessed and thereafter it is native machine language. The \verb|ClangInvokable::init| method creates an instance of its object and stores its address. After that invoking \verb|ClangInvokable::kontrol| every kperiod is just as efficient as invoking Csound's native opcodes.

When the Csound instrument that has created the \verb|clang_invoke| opcode is 
turned off, Csound calls the \verb|ClangInvokable::noteoff| method. At that 
time, the \verb|ClangInvokable| should release any system resources or memory 
that it has acquired.

The \verb|ClangInvokable| instance is then deleted by the \verb|clang_invoke| opcode.

\section{Examples}

\subsection{clang\_hello.csd}

\subsection{clang\_example.csd}

%References
\bibliographystyle{cmj}
\bibliography{EmbeddingClangLLVMinCsound}

\end{document}
